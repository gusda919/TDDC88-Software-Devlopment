\section{Introduction}
\label{sec:introduction}
In chapter one, the purpose and the intended audience of the SRS will be explained. The scope of the software product will also be defined, followed by a list of all the definitions, acronyms and abbreviations used in this document. Lastly, an overview of the entire content of this SRS is presented.  

\subsection{Purpose and intended audience}
The purpose of this document is to provide a detailed description and understanding of the requirements for the ''Digital Akutjournal'' software. To increase the understanding, the document will explain the constraints of the systems and the relevant interfaces. The intended audience for this document is both the customer and other stakeholders for approval and communication purposes. The document is also meant to be used as a foundation for the development team when developing mainly the first draft, but also for support to later versions.

\subsection{Scope}
The software product will be a application for mobile devices developed to be used in Region Östergötlands three emercency rooms located in Motala, Linköping and Norrköping. The application will be used by the health care workers to enable easier data access and data analysing in the ER. This will be done by merging all the patient data from multiple systems and databases into one view that provides graphical overviews of historical data as well as presenting vital parameters and other medical notations. To ensure that each user can quickly access the data that they assess as relevant, the system will allow the user to customize and save their screen settings as well as be customized depending on which user role a worker has. The system will also help to improve the workflow at the ER by notifying the user when new important information is available or a critical change has occurred in a patient's vital parameter. In addition to this, the system will also allow the user to access past health data from earlier visits at the ER. The system will not have any features to support input of data, only functions to support data visualizing since the ER already has working systems to use for the data input into different databases.

\subsection{Definitions, acronyms and abbreviations}

\begin{tabularx}{1.0\textwidth} { 
  | >{\raggedright\arraybackslash}X 
  | >{\raddedright\arraybackslash}X 
  | }
 \hline
 \textbf{Term} & \textbf{Definition}\\
 \hline
 SRS  & System Requirements Specification  \\
 \hline
 ER & Emerceny room \\
 \hline
RÖ & Region Östergötland \\
\hline
User & Health care staff from RÖ working at the ER that interacts with the application \\
\hline
Team & A medical team active at the ER \\
\hline
GUI & Graphical user interface \\
\hline
API & Application programming interface \\\hline
Backend & Server that stores all the data that the application uses. \\ 
\hline
triage-color &  One color that is assigned to the patient after a healthcare worker assesses the patient's condition.\\
\hline
Enhetsöversikt &  A page in the application where the user can get an overview of all patients. (ENG:Page Overview)\\
\hline
Orsakskod & A set of standardized codes that describe the reason for a patient's admission to the hospital. (ENG:Hospital admission codes) \\
\hline
Vätskebalans & The difference between the fluid lost compared to the fluid taken in by a patient. (ENG:Fluid Balance)\\
\hline
Vätskeintag & How much liquid a patient consumes . (ENG:Fluid intake)\\
\hline
Vätskeförlust & The loss of liquid for an patient  (ENG:Fluid loss)\\
\hline
Patientöversikt & A page in the application where the user can get an overview of specific patients and their information. (ENG:Patient Overview)\\
\hline
Referensvärde & A set of values that a user uses to interpret a patient's test results (ENG:Reference value)\\
\hline
Smittsam & (ENG:Contagious)\\
\hline
hemgångsmeddelande & Is a note that the patient gets after getting discharged from the Hospital with relevant medical information. (ENG: Discharge information)\\
\hline
Personnummer & (ENG:National identification number)\\
\hline
Ordinering & the act of administrating medication to a patient of an healthcare worker
 (ENG:Medication Administration)\\
\hline
\end{tabularx}

\subsection{Overview}
This SRS is divided into 3 chapters according to IEEE Std 830-1998. In the first chapter, the application is presented and introduced with the purpose and intended scope of the product, as well as intended audience. The first chapter also contains a list of important definitions, acronyms and abbreviations to ensure that the reader is provided with all information needed to fully understand the content of the SRS. In the second chapter, the context of the application is presented through product perspectives and product functions, user characteristics, constraints, assumptions and dependencies. In the last chapter the most detailed description of the system is presented by providing all the requirements. These are divided into interface, functional, performance, design constraints, software system attributes and others. 