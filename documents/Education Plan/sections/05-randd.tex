\section{Research \& Development}
\label{sec:r&d}
The Research \& Development department is responsible for the development and programming of the product. This includes the configuration manager, architect, deployment manager, technical writer, UX designer and the developers. 

\subsection{General}
\label{sec:r&d:randgeneral}
At the beginning of the project the Research \& Development department focuses on learning and working with prototyping. To get the members up to speed with the MEAN stack, the team leaders provides relevant links for each member to read up on. A document is also created where members can share useful links when doing their own research. In short, in the beginning responsibilites are placed upon each individual to study parts they feel in need to study.

In order to get the development started for everyone, not only the developers, the Architect holds a workshop with the goal of creating a functional development environment for each member of the project. 

To provide a platform for technical questions and to foster a feeling of solidarity a channel exists (see Technical support and helpdesk in Teams). The activity in this channel provides an overview of the problems faced by the project members and will help management in educational purposes (e.g., scheduling workshops, providing helpful educational links, etc.).

\subsection{Developers}
\label{sec:r&d:developers}
Some of the team members had worked with the MEAN stack framework before, and thus it was chosen for this project. Decisions and justifications to why the framework is suitable for this project can be found in the Architecture Notebook.

Both the Architect and the UX designer have previous experience with the MEAN stack, and have the main responsibility of educating the project members when necessary. 


\subsection{UX Designer}
\label{sec:r&d:ux}
The UX designers are responsible for the user experience (UX) of the application. The lead UX designer has a background in UX, user testing and Figma. The document "Länkar och Tips 2021" lays the foundation for the UX designers to work with and educate themselves with if needed. Additionally, with the lead UX designer's previous experience and self-education within the subject, she is able to quickly identify areas of risk and address them by further education or educating fellow UX designers. 

To relieve the lead UX designer, the configuration manager has taken a secondary role as a UX designer, since the workload was too high in iteration 1. Additionally, management have decided to have a representative each from the analyst team and test team to report to the lead UX designer. This way the lead UX designer can share insights with both groups while the representatives also work as a sounding board. 

\subsection{GitLab}
\label{sec:r&d:git}
The Configuration manager, Deployment manager, and the Quality Coordinator are responsible for GitLab structuring and education. To educate themselves, they have studied the provided links about GitLab by the course examiner, amongst other sources. To spread what they've learnt, the information has been translated into guidelines listed in the Software Quality Assurance Plan and a Readme-file found in the root of the development project. 
