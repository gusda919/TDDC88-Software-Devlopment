\section{Education plan}
In this section, general plans shall be documented. Following that, specific plans for the different teams shall be detailed. 

\subsection{General plans}
The primary educational source shall be the course lectures, labs, exercises, and finally, the examination. To complement these, members shall follow the educational processes described in section 2. 

In our organization, we use a few tools to help us communicate with each other. The primary tool is Microsoft Teams, where communication is handled in channels. The main channel is used for company-wide information, meetings, and the prominent place to store files, while smaller teams have their own channels where more specific information is shared. Everyone has access to every channel to make the information as transparent as possible.

For structuring work, company members shall use Trello to create Kanban-boards. Tasks are put up on the list, and people can either assign themselves or be assigned a task to work on, which is a concept everyone is familiar with when starting the project. No education is needed for these communication tools since everyone is already familiar with them.

In order to centralize the information and minimize the number of tools used by the project members, it is decided in Iteration 1 that the Kanban-boards will be migrated from Trello to GitLab. A workshop shall complement this migration to make it easier for project members. The Configuration Manager, Deployment Manager, and the Quality Coordinator are responsible for GitLab structuring and education. To educate themselves, they shall study the provided links about GitLab by the course examiner, amongst other sources. To spread what they have learned, the information is translated into guidelines listed in the Software Quality Assurance Plan and a README file found at the root of the development project. 

\subsection{Product and sales}
The Product \& Sales department (P\&S) consists of four smaller groups of people with more focused work, which are the analysts, the testers, quality, and a group of managers, including the Project Manager, Product Manager, and the Line and Process Manager. Their education plan is described below.

\subsubsection{Analysts}
To get the proper knowledge to perform their tasks, the analyst's education is based on what we learn in the lectures given in this course about these topics. The lab series in the course also has a lab regarding this topic, which will also be used to gain knowledge.

Continuous dialogue with the customer and visits to the emergency departments will give the analyst team a clear idea of what the customer wants and how to best solve the problem at hand. These will be key for developing use cases and requirements. 

\subsubsection{Testers}
Testers will start to learn about testing by studying the course material presented in lectures and labs to get a basic understanding of what their role is responsible for. The Karma framework for automated tests is chosen together with Selenium to test the application's front-end. The testers will perform self-studies to get the required knowledge of these tools. 

\subsubsection{Quality}
The lecture material in the course regarding software life-cycle models and methods, software metrics, software reviews, and software quality management is the foundation of the quality coordinator's education within software quality. The lab exercises and lecture exercises with quality themes are also beneficial to the learning. 

\subsubsection{P\&S managers}
The managers shall use the course material to educate themselves, and find complementary information when needed. As the managers have little experience in management, they shall help each other as much as possible. The company CEO can act as a good source of information and sounding board. 

\subsection{Research and development}
The Research \& Development department (R\&D) is responsible for developing and programming the product. This includes the Configuration Manager, Architect, Deployment Manager, Technical Writer, UX Designer, and Developers. 

\subsubsection{Developers and UX designers}
As the architect and the lead UX designer have previous experience with the MEAN stack framework (chosen framework for the development), they shall have the primary responsibility of educating fellow developers and UX designers. 

Developers shall, in the beginning, focus on self-education by reading documents shared by the architect. The developers shall share educational content in a separate documents (found \href{https://liuonline.sharepoint.com/sites/TDDC88_2021_C1/Delade\%20dokument/Forms/AllItems.aspx?RootFolder=\%2Fsites\%2FTDDC88\%5F2021\%5FC1\%2FDelade\%20dokument\%2FGeneral\%2FR\%26D&FolderCTID=0x012000EFD53ADC1DFA3C4F9DA49133CF99E2A6}{here}). Developers shall make use of booked rooms on scheduled course time to sit together and pair-program when needed. This way, members can educate each other and widen each other's perspectives. 

The lead UX designer shall make use of the document "Länkar och Tips 2021" which lays the foundation for the UX team to work with and educate themselves with. The UX team shall share links in separate documents (same as the development team, link found above). 

\subsubsection{R\&D managers}
The managers of the (R\&D) department shall do self-education with the help of course material, and other helpful content when needed. The managers shall make use of the supervisors for additional education. 