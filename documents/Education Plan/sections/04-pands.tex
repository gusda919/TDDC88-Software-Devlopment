\section{Product \& Sales}
\label{sec:pands}

The Product \& Sales department consists of three smaller groups of people with more focused work, which are the analysts, the testers and the quality assurance. Their education plan is described below.

\subsection{Analysts}
At the start of the project, the analysts will gather requirements about the product we will create. They will also make user stories and use cases to better explain what the product will do. To get the right knowledge about this, their education is based on what we learn in the lectures given in this course about these topics. The lab series in the course also has a lab regarding this topic, which will also be used to gain knowledge.

Continuous conversation with the customer and visits at the emergency departments will give the analyst team a clear idea of what the customer wants and how to best solve the problem at hand. These will be key for developing use cases and requirements. To develop the use cases, a workshop is held to make sure that every use case is developed in a pre-determined way. To do this a template for use cases is used during the workshop. After the requirements are developed, they are ranked based on how important they are to the product.

\subsection{Testers}
Testers are responsible to test the product to make sure it works as intended. They will start to learn about testing by studying the course material presented in lectures and labs to get a basic understanding about what their role is responsible for. The Karma framework for automated tests is chosen together with Selenium for testing the front-end of the application. The testers will perform self studies to get the required knowledge of these tools. 

To get a good start with testing in the project a workshop is held where all the testers attend to synchronize their work and the tools that is going to be used. The different tools that are going to be used is set up so that all the testers has a working version and knows how to use it. Karma and Jasmine was evaluated in the workshop to see which framework was going be best to use in the project.

\subsection{Quality}
Quality is responsible for ensuring that the product fulfills the quality standards we have put on it. The lecture material in the course regarding software life-cycle models and methods, software metrics, software reviews and software quality management is the foundation of the education within software quality. The laboration and lecture exercises with quality themes are also beneficial to the learning. Adding to this, walkthroughs and documentation of GitLab (quality) features are reviewed.



