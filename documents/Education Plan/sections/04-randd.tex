\section{Research \& Development}
\label{sec:r&d}
The Research \& Development department is responsible for the development and programming of the product and includes the configuration manager, architect, deployment manager, technical writer, UX designer and the developers. 

\subsection{General}
\label{sec:r&d:randgeneral}
At the beginning of the project the Research \& Development department focused on learning and working with prototyping. With regards to the experience in the department at the beginning of the project and the requirements of the project, it was decided that the MEAN stack framework would be used in developing the product later on. To get the members up to speed the team leaders provided relevant links for each member to read up on. A document was also created where members could share useful links when doing their own research. In short, in the beginning responsibilites were placed upon each individual to study parts they felt in need to study.

In order to get the development started for everyone, not only the developers, the Architect held a workshop with the goal of creating a functional development environment for each member of the project. 

\subsection{Developers}
\label{sec:r&d:developers}
Some of the team members had worked with the MEAN stack framework before, and thus it was chosen for this project. Decisions and justifications to why the framework i suitable for this project can be found in the Architecture Notebook.


\subsection{UX Designer}
\label{sec:r&d:ux}
The UX designer is responsible for the user experience (UX) of the application. The document "Länkar och Tips 2021" laid the foundation for the UX designer to work with. 

In iteration 1 the lead UX designer, Olivia Shamon, realized that the work with UX was too much for one person to handle. To relieve the lead UX designer, Albin Ambrosius has taken a secondary role as a UX designer in addition to being the configuration manager. Additionally, management have decided to have a representative each from the analyst team and test team to report to the lead UX designer. This way the lead UX designer can share insights with both groups while the representatives also work as a sounding board. 

\subsection{Git}
\label{sec:r&d:git}

