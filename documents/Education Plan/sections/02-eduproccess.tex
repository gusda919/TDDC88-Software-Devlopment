\section{Educational process}
\label{sec:eduproc}

The company works with different educational processes to educate and spread knowledge to the members of the company. These processes can be divided into three categories listed below:

\begin{itemize}
    \item Workshops – held by someone with expertise when the company introduces, for example, a new process and management feel like the company members will benefit from a walk-through of the process. Employees can also request workshops if the necessary knowledge to perform certain tasks is broader than the existing knowledge. Workshops can also be held as a brainstorming session to gain input from other employees.
    \item Informational documents – members of the company can write and share informational documents to guide other members on how to perform certain tasks. An example of an informational document is the readme-file found on Gitlab, which outlines the process of working on Git.
    \item Gathering and sharing educational content – members of the company shall gather and share educational content if and when needed, e.g., the test leader can find content, like relevant websites or videos relating to testing, and share with the testers to read up on. 
\end{itemize}
