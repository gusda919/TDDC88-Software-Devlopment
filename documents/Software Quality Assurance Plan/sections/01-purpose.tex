\section{Purpose}
\label{sec:purpose}

The purpose of this Software Quality Assurance Plan is to establish the processes and responsibilities required to implement effective quality assurance functions for the project.

The Software Quality Assurance Plan will provide the framework necessary to ensure a consistent approach to software quality assurance throughout the project life cycle. It defines the approach that will be used by members of the company to monitor and assess software development processes to provide objective insight into the quality of the software and the product. The systematic monitoring of the product and processes will be evaluated to ensure they meet requirements. 

\subsection{Scope}
This plan covers Software Quality Assurance activities of the project and will be updated regularly.

\subsection{Project description}
The project's goal is to develop a tool that visualizes and compiles information from patients at Region Östergötland hospital's emergency departments. The information comes from the various computer systems that Region Östergötland has available today. The tool should compile information in an easy and clear way. The compiled information must be presented but does not have to include input solutions.

The product will be developed during four iterations. The project will continue until December 16, 2021, when version 1.0 will be presented.

\subsection{Estimated resources}
\begin{table}[!ht]
\centering
\begin{tabular}{ | l | c |}
    \hline
Task & Estimated hours\\
\hline
Education & 12\\
Developing Quality Processes & 25\\
Documenting Quality Processes & 25\\
Monitor Documents & 14\\
Testing Quality Assurance & 8\\
Software Quality Assessments & 8\\
Supervisor Meetings & 8\\
\hline
\end{tabular}
\end{table}
\noindent The estimated resources presented in the table above applies for the Quality Coordinator of the project. Education covers both learning about quality assurance and software metrics as well as internal workshops held within the company. Developing Quality Processes and Documenting Quality Processes are the two biggest posts since the company starts from scratch. Work with these tasks is mainly about developing guidelines for the company and documenting them in the Software Quality Assurance Plan and on the company's GitLab. These three posts are the main focus in the first half of the project. More information about Monitoring Documents can be found under subsection \ref{subsec:monitor-docs}, for Testing Quality Assurance see section \ref{sec:testing} and Software Quality Assessments see section \ref{sec:sqa}.

%\subsection{Quality objectives}
%\textcolor{red}{Develop this! Goals related to testing?}
%\href{https://www.nqa.com/en-gb/resources/blog/may-2016/top-quality-objectives}{https://www.nqa.com/en-gb/resources/blog/may-2016/top-quality-objectives}

%Reach a certain score on SUS-tests, more good comments than bad comments in user testing, improve process..?