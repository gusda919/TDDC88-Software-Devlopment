\section{Development}
This section is intended to address guidelines for the development and quality assurance of code that will result in the end product.

The project is set up with TypeScript. The client (Angular), server (Node.js) and database (MongoDB) run in three seperate Docker containers and are orchestrated with Docker-compose.

\subsection{Code conventions}
As the project is set up with TypeScript (TS), \href{https://google.github.io/styleguide/tsguide.html}{Google's TypeScript Style Guide} must be followed for all TS code.

Names for functions and variables must be written with camelCase and be self-explanatory. Names generated by Angular should not be changed. \emph{Conventions for HTML and CSS is to be determined}.

\subsection{Workflow}
The project uses a form of feature branch workflow. Feature branches are merged with a development branch, never directly to the main branch. Once the code has passed testing on the develop branch, it can be merged into the main branch by a authorized manager.

\emph{Figure of workflow will be added in an upcoming version.}

\subsection{Software reviews}
In addition to regular discussion of code between developers, software reviews take place in connection with each merge from feature branches to the development branch. The software review must follow a set protocol in GitLab \emph{(will be attached in this docuement in an upcoming version)}. In merge requests, new code must be described as well as what changes have been made and the justification for these.

The reviews are done by a person who has not developed the code himself within two working days of when the merge request was posted. For critical features, the review must be done by a developer with at least competence level 4 \emph{(TBD)}.

\subsection{Bug-tracking system}
GitLab is used for bug-tracking. Bugs will be listed, classified and get unique ID as issues in \href{https://gitlab.liu.se/tddc88-company-1-2021/deploy/-/issues}{the Issues tab of the company GitLab}. \emph{Further guidelines for bug-tracking will be found on the company Gitlab.}

\subsection{Version handling}
GitLab is used for version handling. Work needs to be commited at least daily when working on features but the goal is to commit after each working addition to the code. Commits must be tagged with what kind of commit it is \emph{(see upcoming guidelines on GitLab)} and the commit message should include the ID for the requirement or issue being worked on and the functions that have been changed or added.

\subsection{Documentation}
The Technical Writer and the involved developers document the code produced at module level. The documentation includes descriptions of which inputs are handled, which functions are used, which inputs are given and how they work together with the rest of the system.
