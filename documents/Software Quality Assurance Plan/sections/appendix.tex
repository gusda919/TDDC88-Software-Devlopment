\appendix
\label{sec:appendix}

\section{\LaTeX{} Examples}
The following examples should help you to write your document using \LaTeX{}. You'll find here the examples of tables, figures, citations and references. For other features of \LaTeX{}, see tutorials on \href{https://www.overleaf.com/learn}{\textbf{Overleaf}} or use this \href{https://wch.github.io/latexsheet/}{\textbf{cheatsheet}}. To work with this template, download its entire folder (including /sections, /bibliography and /figures), and run your \LaTeX{} editor like \href{https://www.overleaf.com}{\textbf{Overleaf}}.


\subsection*{Example Citation}
Example of citation: \cite{Smith_2013} and \cite{Smith_2012}. 


\subsection*{Example References}
%Example of table reference: see Table \ref{tab:example}. \\
Example of equation reference: see Equation \eqref{eq:emc}. \\
%Example of reference to Section \ref{sec:methods}. \\
%Example of reference to Subsection \ref{sec:dataset:subsection}. \\
Example of figure reference: see Figure \ref{fig:example}.\\
Example of subfigure reference: see Figure \ref{fig:multiple:example11}.\\


\subsection*{Example list}
\begin{itemize}
\item Bullet point one
\item Bullet point two
\item Nested list items:
\begin{itemize}
\item Nested item one
\item Nested item two
\end{itemize}
\end{itemize}

\subsection*{Enumerations}
\begin{enumerate}
\item Numbered list item one
\item Numbered list item two
\item Nested list items:
\begin{enumerate}
\item Nested item one
\item Nested item two
\end{enumerate}
\end{enumerate}


\subsection*{Example Table}

\begin{table}[h] 
\centering
\begin{tabular}{l l l}
\hline
\textbf{Treatments} & \textbf{Response 1} & \textbf{Response 2}\\
\hline
Treatment 1 & 0.0003262 & 0.562 \\
Treatment 2 & 0.0015681 & 0.910 \\
Treatment 3 & 0.0009271 & 0.296 \\
\hline
\end{tabular}
\caption{Table caption}
\label{tab:example}
\end{table}



\subsection*{Example Equation}
Equations within the text: $e = mc^2$. Equation with label on its own line:
\begin{equation} \label{eq:emc}
    e = mc^2
\end{equation}




\subsection*{Example Figures}

\begin{figure}[ht]
    \centering\includegraphics[width=0.4\linewidth]{placeholder}
    \caption{An example of simple figure.}
    \label{fig:example}
\end{figure}

\begin{figure}[ht]
    \centering
    \begin{subfigure}[t]{0.4\textwidth}
        \centering\includegraphics[width=1\linewidth]{placeholder}
        \caption{An example of multiple figures in one frame.}
        \label{fig:multiple:example11}
    \end{subfigure}
    %
    \begin{subfigure}[t]{0.4\textwidth}
        \centering\includegraphics[width=1\linewidth]{placeholder}
        \caption{Next subfigure.}
        \label{fig:multiple:example12}
    \end{subfigure}
    %
    \\
    \begin{subfigure}[t]{0.4\textwidth}
        \centering\includegraphics[width=1\linewidth]{placeholder}
        \caption{Subfigure on another line.}
        \label{fig:multiple:example21}
    \end{subfigure}
    %
    \begin{subfigure}[t]{0.4\textwidth}
        \centering\includegraphics[width=1\linewidth]{placeholder}
        \caption{Yet another subfigure.}
        \label{fig:multiple:example22}
    \end{subfigure}
    \caption{More figures in appendix.}
    \label{fig:multiple}
\end{figure}