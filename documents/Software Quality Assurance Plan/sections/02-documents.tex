\section{Documents}
This section lists the regulatory documents used in the project and the plan for Quality Assurance related to documents.

\subsection{Regulatory documents}
The Quality Assurance of Documents apply to the six regulatory documents listed below. They are available in the Output documents folder on MS Teams.
\begin{itemize}
\item Architecture Notebook
\item Customer Requirements Specification
\item Education Plan
\item Project Plan
\item Software Quality Assurance Plan \emph{(this document)}
%\item \textcolor{red}{Standard Operating Procedures}
\item Test Plan
\end{itemize}

\subsection{Plan to monitor documents}
The Quality Coordinator shall monitor the regulatory documents and make they follow the company guidelines. This Quality Assurance work is done on a weekly basis. The Quality Coordinator will check if the guidelines are followed and if essential content is missing. Remarks from the review is left as comments in the document editor and after the review is over, the Quality Coordinator sends a summary of the review to the author of the latest version of the document.
%\emph{Lägg till beskrivning och schema över när/hur ofta dokument kommer granskas av Erik?}
%3/11, 16/11, 30/11, 8/12

\subsection{Guidelines}
\emph{For version 1.2 of this document, the guidelines and processes regarding documents is found in the Project Plan (as of 2021-11-04 in section 5 - Processes).}  
%\begin{itemize}
%\item Regulatory document shall be written in LaTeX, preferably with the help of Overleaf. The documents are made available to everyone in the company through link sharing, editing links to all documents are among the company's files in Teams. 
%\item Version control of the regulatory documents is made via \href{https://gitlab.liu.se/tddc88-company-1-2021/deploy/-/tree/develop}{GitLab}. Each document has its own directory in the documents folder of the develop branch. Version control shall be carried out regularly, at least when each new version of the document is published.
%\item To ensure traceability: When refering to requirements, issues or other artifacts in documentation, ID of the artifact in question must be included. When documenting actions, such as decisions or changes, pointers to relevant documentation must be included.
%\item \textcolor{red}{For major changes to be made to requirements, a change request \emph{(process TBD)} must be accepted by the product owner and reviewed by a company member with knowledge of the subject. Major changes must be documented in accordance with \emph{(process TBD)} where the requirements and issues involved are clearly referred to.}
%\item In order to publish new versions of regulatory documents, a company member with knowledge of the subject must have reviewed the updated document.
%\item When changes are made to published documents, version number, author, significant changes between versions, reviewer and date must be noted in the version table.
%\end{itemize}