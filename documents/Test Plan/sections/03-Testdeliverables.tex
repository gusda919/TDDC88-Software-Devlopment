\section{Test deliverables}
This chapter regards deliverables created for customer, CEO, company and/or testers.

% \noindent In the end of each iteration we intend to have a meeting with the customer where they will do a System Usability Scale (SUS) test with the Concurrent Think Aloud (CTA) protocol. Every iteration is roughly two weeks so we are planning to do these test four times with the customer during the project. This is to test the new features that has been developed during the iteration that has been so that the customer can give feedback to what they like and don't like. The feedback from the customer will then be delivered to the developers and UX-designers so that they can continue to improve the features and modules. This feedback is summarized by the testers, but the original meetings are recorded if possible and uploaded to enable the possibility to go back and analyze a specific meeting. All results from the meetings will be saved to the files section of the company "Testing" Teams channel. \newline
\noindent At the end of each iteration, we intend to have a meeting with the customer where they will do a System Usability Scale (SUS) test with the Concurrent Think Aloud (CTA) protocol. Every iteration is roughly two weeks, so we are planning to do these tests four times with the customer during the project. This is to test the new features that has been developed during the iteration that has been so that the customer can give feedback on what they like and do not like. The feedback from the customer will then be delivered to the developers and UX designers so that they can continue to improve the features and modules. This feedback is summarized by the testers, but the initial meetings are recorded if possible and uploaded to enable the possibility to go back and analyze a specific meeting. All results from the meetings will be saved to the files section of the company "Testing" Teams channel. \newline

% \noindent The first customer meeting did not feature a SUS-test with a CTA protocol as the application was deemed too bare-bones to result in any meaningful feedback from a customer. Instead, a customer meeting was conducted where design ideas from the Figma protype and the application were evaluated on a detail level and written down. This interview was done with nurses and IT-staff from the customer. The notes from the meeting were then passed to the UX-team for evaluation.\newline
\noindent The first customer meeting did not feature a SUS-test with a CTA protocol as the application was deemed too bare-bones to result in any meaningful feedback from a customer. Instead, a customer meeting was conducted where design ideas from the Figma prototype and the application were evaluated on a detailed level and written down. This interview was done with nurses and IT staff from the customer. The notes from the meeting were then passed to the UX-team for evaluation.\newline

%infoga lista på uppgifter kund gör här när vi har en sådan
% \noindent System testing is going to be performed for each merge request and bugs found will be commented in code and as a comment in the merge request text fields. The creator of the merge request are then notified by the automated mail sent to their Gitlab accounts of the discovered bug. If \newline
\noindent System testing is going to be performed for each merge request; any bugs found will be commented in code and as a comment in the merge request text fields. The creator of the merge request is then notified by the automated mail sent to their GitLab accounts of the discovered bug. If \newline

% \noindent The final system testing will be performed and will result in a excel file with all the requirements and how they are tested. The file will have information if the requirement is completed or not. The document name is  \textit{Testing\_TestCases} and can be found in the output folder.   
\noindent The final system testing will be performed and will result in an excel file with all the requirements and how they are tested. The file will have information on whether the requirement is completed or not. The document name is  \textit{Testing\_TestCases} and can be found in the output folder. 
 
\noindent The test plan is created to standardize and inform of the testing procedure. It is written by the testers. It is published both in the GitLab repository as well as the Teams output folder for the CEO and company to be informed about how testing is conducted. When a testing procedure is changed or created, it is written down into the test plan by testers, so no procedure goes undocumented.

\subsection{Milestones}
\subsubsection{Pre-study}
% \noindent In the pre-study phase of the project there will be no code to test. The testers will spend their time in preparation for the upcoming iterations. This preparation consisted of deciding upon which testing tools that were to be used during the project lifespan and how the process of testing is going to be implemented in the project. These processes aimed to answer the question of how different tests were going to be performed, what defines a passed test and how bugs are to be reported. The first part of the test plan is going to be written during the pre-study phase. A workshop is going to be held to understand which tools that we should use during this project. 
\noindent In the pre-study phase of the project, there will be no code to test. The testers will spend their time in preparation for the upcoming iterations. This preparation consisted of deciding upon which testing tools were to be used during the project lifespan and how the process of testing will be implemented in the project. These processes aimed to answer how different tests were going to be performed, what defines a passed test, and how bugs are to be reported. The first part of the test plan is going to be written during the pre-study phase. A workshop will be held to understand which tools we should use during this project.
%Further self-studies is going to be needed to get a good understand of software testing.

\subsubsection{Iteration 1}
The main focus is to set up a plan for how the testing should be performed. The plan should be specified in the test plan. At the end of the iteration, a meeting with the customer is going to be held to get further opinions from the customer to refine our requirements. In this iteration, the usability testing is specified how it should be performed.
\subsubsection{Iteration 2}
% The goals for the second iteration are to refine the test plan to better find bugs and improve our process. We are going to continue to test new code. In this stage the development has probably increased in speed and therefore more time needs to be set aside for actual testing. In this stage the pipeline is going to be created to have automated test. At the end of the iteration, the tester will perform usability testing with the customer to understand how usable our application is. To be able to conduct these test, a plan for usability test needs to be created. Each tester will perform a test on their own and write a test report of how well the test went and then all test report is summarized to a single report.
The goals for the second iteration are to refine the test plan to find bugs better and improve our process. We are going to continue to test the new code. In this stage, the development has probably increased in speed, and therefore, more time needs to be set aside for actual testing. In this stage, the pipeline is going to be created to have the automated test. At the end of the iteration, the tester will perform usability testing with the customer to understand how usable our application is. To be able to conduct these tests, a plan for usability tests needs to be created. Each tester will perform a test on their own and write a test report of how well the test went, and then all test report is summarized into a single report.
 
\subsubsection{Iteration 3}
% In this iteration we should further refine our testing procedures and continue testing all new developed features. We should also further enhance our documents and usability testing. If the pipeline is not done at this stage it is important to finalize it now.  
In this iteration, we should further refine our testing procedures and continue testing all newly developed features. We should also further enhance our documents and usability testing. If the pipeline is not done at this stage, it is crucial to finalize it now. 

\subsubsection{Iteration 4}
% In the last iteration the test team should make sure that the final acceptance testing is performed. In this iteration the test team should check that every requirement that's created are verified.     
% Setting up final acceptance testing with customer for the final product and performing the test. Further continue to test merge requests. In addition to the final system and acceptance testing we will schedule one more occasion for system and acceptance testing if the customer doesn't approve the product during the first acceptance test. There will be approximately one week between these occasions so that the developers have time to add or change features and we can test and do a final system testing before the acceptance testing. \\
In the last iteration, the test team should ensure that the final acceptance testing is performed. In this iteration, the test team should check that every created requirement is verified.     
We set up final acceptance testing with the customer for the final product and perform the test. Further, continue to test merge requests. In addition to the final system and acceptance testing, we will schedule one more occasion for system and acceptance testing if the customer does not approve the product during the first acceptance test. There will be approximately one week between these occasions so that the developers have time to add or change features, and we can test and do a final system testing before the acceptance testing. \\


\noindent To further concrete the test plan with the most important dates at the end of this project, we have set up milestones dates for these events:\\
\noindent 22-Nov - Freeze branch to be tested \& review task for acceptance testing\newline
\noindent 23-Nov - Finish test plan v2.0 for reviewing\newline
\noindent 24-Nov - Totally finished test plan v2.0\newline
\noindent 25-Nov - Summarize test reports (from SUS-tests)\newline
\noindent30-Nov - Test plan v3 ready for review\newline
\noindent1-Dec - Totally finished test plan v3.0 \& preparation for Final system testing\newline
\noindent2-Dec - Final system testing\newline
\noindent3-Dec - Final acceptance testing\newline
\noindent7-Dec - Ev. Final system testing\newline
\noindent8-Dec - Ev. Final acceptance testing\newline



