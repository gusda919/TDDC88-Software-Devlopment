\section{Test deliverables}
This chapter regards deliverables created for customer, CEO, company and/or testers.

\noindent In the beginning of each iteration we intend to have a meeting with the customer where they will do a System Usability Scale (SUS) test with the concurrent think alound (CTA) protocol. Every iteration is roughly two weeks so we planned to do these test four times with the customer during the project. This is to test the new features that have been developed during the iteration that has been so that the customer can give feedback to what they like and don't like. The feedback from the customer will then be delivered to the developers and UX-designers so that they can continue to improve the features and modules. This feedback is summarized by the testers, but the original meetings are recorded if possible and uploaded to enable the possibility to go back and analyze a specific meeting. All results from the meetings will be saved to the "Testing" Teams channel. \newline

\noindent The first customer meeting did not feature a SUS-test with a CTA protocol as the application was deemed too bare-bones to result in any meaningful feedback from a customer. Instead, a customer meeting was conducted where design ideas from the Figma protype and the application were evaluated on a detail level and written down. This interview was done with nurses and IT-staff from the customer. The notes from the meeting were then passed to the UX-team for evaluation.\newline

%infoga lista på uppgifter kund gör här när vi har en sådan
\noindent Unit and integration testing is done at each merge request and bugs found will be commented in code and as a comment in the merge request text fields. The creator as assignees of the merge request are then notified by the automated mail sent to their Gitlab accounts of the discovered bug.\newline

\noindent The test plan is created to standardize and inform of the testing procedure. It is written by the  the quality coordinator (namely \autoref{chap:qualityobj}) and the testers. It is published both in the Gitlab repository as well as the Teams output folder for CEO and company to be informed about how testing is conducted. When a testing procedure is changed or created, it is written down into the test plan by testers so no procedure goes undocumented.

