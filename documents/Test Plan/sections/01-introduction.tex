\section{Introduction}
In this document an explanation about our test strategies, process, workflow and methodologies used in the project is given.  The test plan will describe the structure of our testing and how they are incorporated in the process of developing the product. Every type of test that we will perform is going to be further explained into detail. The reason why we will perform certain tests at which stage in the project will also be explained. This is going to be used for all the testers to work in the same way and to have a standard within the company, this test plan will be read by new testers for learning how to work with the testing. Therefore there will be further explanation of how to handle bugs found during testing, what tools that are used and when a test is considered complete. \newline 

\noindent We follow the V-model which is a model that has a testing phase parallel to the development phases. This model is also called Verification and Validation model, it is an extension of the waterfall model as it is also a sequential model. There is a testing phase parallel to each development stage. This means that it includes testing for the requirements, system design, module testing and the implementation. The advantage of using this model is that it is simple and easy to follow and when the requirements are well defined, it also works well for smaller projects like this.\newline

\noindent The development will be performed agile which mean that it is performed iteratively. Therefore, the testing is also done iteratively. The V-model will be done iterative and the different levels of testing will be performed for each iteration. This will iteratively enhance the testing because retrospectives of the sprints will be done after each iteration that will highlight the pros and cons of the testing. This will also strengthen the customer relation because acceptance testing with the customer will be held in each sprint. The customer can validate our requirements, if we are going in the right direction or if they have changed their needs. \newline

\noindent The following test plan is derived from the test plan template created by Thomas Hamilton for Guru99. A link to the template can be found \href{https://www.guru99.com/test-plan-for-project.html}{here}. 

\subsection{Scope}
The scope of the test plan is defined below. This includes what will be tested (In Scope) and the parts that will not be tested (Out of Scope).  

\subsubsection{In Scope}
In accordance with the V-model, testing is to be conducted in unit, module, system and acceptance level testing. \newline

\noindent The customer wants us to focus on the usability of the system and how to visualise data in a satisfying way.  Therefore the testing will focus on that and measure if the developed application achieves requested levels. The testing will also focus on the functionality of the system so that the intended units and integration between units works as expected. The testing in this project will include Unit testing which tests the individual units, module testing which tests the integration between the different units, system testing which tests the integration between different modules and also acceptance testing that verifies the requirements. All these tests and how we are going to perform them in the project is going to be explained in the test plan. The main focus of the testing is to ensure that the requirements are met in the finished product. 

\subsubsection{Out of Scope} 
Due to the limited implementation levels of the project the test plan will diverge from the V-model by not conducting tests at the final level, maintenance. \newline

\noindent The customer does not want us to focus on how to store data and database management and therefore such tests will not be conducted. Because the storing of data is not important  for this project, security will neither be tested during this project. The testing of the database could otherwise have been done by doing Structural Database Testing which validates the elements that are inside the database that are stored and cannot be accessed or changed by the end users. 

%\subsection{Quality Objectives}
%\label{chap:qualityobj}
    % Will be included in next version of the SQAP

\subsection{Roles and Responsibilities}

\textbf{Testers}\newline
\noindent In the first part of the project (iteration 1 and 2) every tester is part of a cross-functional team (CFT) and is responsible for performing tests for their CFT individually related to the unit, module and system testing. When merge requests are created from a teams branch to the develop branch the tester that is in that team is responsible for the testing of that merge request. \newline

\noindent From interaction 3 and forward there will be a specific team focused on testing, this means that the testers will work more together and are not responsible for specific merge requests. Instead the testers will take on responsibility for a merge request by assigning themselves for testing that merge request in Gitlab. \newline

\noindent When a bug is found the tester is responsible for reporting it to the developer. This is done through not accepting the merge request and writing comments explaining what the bug is and the suspected underlying issue. If there is a bigger problem that will take longer than two hours to fix a new git issue will be made for that specific problem. \newline

\noindent \textbf{Test leader}\newline
\noindent The test leader is responsible for  the  acceptance  testing  and  will  coordinate  this  with  the  help  of  all  the  testers.  The  test leader will also be the contact person for non CFT-related inquiries for testing procedures. The test leader will also be responsible for setting up meetings with the customer for performing the acceptance testing.
