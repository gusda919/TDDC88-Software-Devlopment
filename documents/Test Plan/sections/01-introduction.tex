\section{Introduction}
% In this document an explanation about our test strategies, process, workflow and methodologies used in the project is given. The test plan will describe the structure of our testing and how they are incorporated in the process of developing the product. Every type of test that we will perform is going to be further explained into detail. The reason why we will perform certain tests at which stage in the project will also be explained. This is going to be used for all the testers to work in the same way and to have a standard within the company, this test plan will be read by new testers for learning how to work with the testing. Therefore there will be further explanation of how to handle bugs found during testing, what tools that are used and when a test is considered complete.
In this document, an explanation of our test strategies, process, workflow, and methodologies used in the project is given. The test plan will describe the structure of our testing and how they are incorporated in the process of developing the product. Every type of test that we will perform is going to be further explained in detail. The reason why we will perform certain tests at which stage in the project will also be explained. This is going to be used for all the testers to work in the same way and to have a standard within the company; this test plan will be read by new testers for learning how to work with the testing. Therefore there will be further explanation of how to handle bugs found during testing, what tools are used and when a test is considered complete.
\newline 

% \noindent We base our testing procedure on the V-model. This model is also called Verification and Validation model, it is an extension of the waterfall model as it is also a sequential model. There is a testing phase parallel to each development stage. This means that it includes testing for the requirements, system design, module testing and the implementation. The advantage of using this model is that it is simple and easy to follow and when the requirements are well defined, it also works well for smaller projects like this.
\noindent We base our testing procedure on the V-model. This model is also called the Verification and Validation model; it is an extension of the waterfall model as it is also a sequential model. There is a testing phase parallel to each development stage. This means that it includes testing for the requirements, system design, module testing, and implementation. The advantage of using this model is that it is simple and easy to follow, and when the requirements are well defined, it also works well for smaller projects like this.
\newline

\noindent The development will be performed agile which means that it is performed iteratively. Therefore, the testing is also done iteratively. This will enhance the testing because retrospectives of the sprints will be done after each iteration that will highlight the pros and cons of the testing. This will also strengthen the customer relation because acceptance testing with the customer will be held after each sprint. The customer can validate our requirements, if we are going in the right direction or if they have changed their needs.
\newline

\subsection{Scope}
The scope of the test plan is defined below. This includes what will be tested (In Scope) and the parts that will not be tested (Out of Scope).  

\subsubsection{In Scope}
% Testing is to be conducted in system and acceptance level testing. The customer wants us to focus on the usability of the system and how to visualise data in a satisfying way.  Therefore the testing will focus on that and measure if the developed application achieves requested levels. The testing will also focus on the functionality of the system. This means that most of the testing in this project is related to the System testing which tests the integration between different modules and also acceptance testing that verifies the requirements. These tests and how they are going to be performed in the project is going to be explained later in this document. The main focus of the testing is to ensure that the requirements are met in the finished product. 
Testing is to be conducted in system and acceptance level testing. The customer wants us to focus on the system's usability and satisfyingly visualize data.  Therefore the testing will focus on measuring if the developed application achieves the requested levels. The testing will also focus on the functionality of the system. This means that most of the testing in this project is related to System testing which tests the integration between different modules, and acceptance testing that verifies the requirements. These tests and how they are going to be performed in the project are going to be explained later in this document. The main focus of the testing is to ensure that the requirements are met in the finished product. 
\subsubsection{Out of Scope} 
Due to the limited implementation levels of the project, the test plan will diverge from the V-model by not conducting tests at the final level, maintenance. It also diverges from a full V-model by not conducting module testing and unit testing. This is due to the focus on usability and displaying data. This will result in a low amount of components needed to be integrated during the project together with the basic level of functionality and output of the implemented functions. Unit and module level testing can be time-consuming, and we have a limited amount of time in the project is another reason why we will not conduct testing on these levels. \newline

% \noindent The customer does not want us to focus on how to store data and database management and therefore tests on the application backend will not be conducted. Because the storing of data is not important  for this project, security will neither be tested during this project. The testing of the database could otherwise have been done by doing Structural Database Testing which validates the elements that are inside the database that are stored and cannot be accessed or changed by the end users. 
\noindent  The customer does not want us to focus on how to store data and database management, and therefore tests on the application back-end will not be conducted. Because storing data is not crucial for this project, security will neither be tested during this project. The testing of the database could otherwise have been done by doing Structural Database Testing which validates the elements that are inside the database that is stored and cannot be accessed or changed by the end-users. 

%\subsection{Quality Objectives}
%\label{chap:qualityobj}
    % Will be included in next version of the SQAP

\subsection{Roles and Responsibilities}

\textbf{Testers}\newline
% \noindent In the first part of the project (iteration 1 and 2) every tester is part of a cross-functional team (CFT) and is responsible for performing tests for their CFT individually related to the system testing. When a merge requests are created from a teams branch to the develop branch the tester that is in that team is responsible for the testing of that merge request.

\noindent In the first part of the project (iteration 1 and 2), every tester is part of a cross-functional team (CFT) and is responsible for performing tests for their CFT individually related to the system testing. When merge requests are created from a team's branch to the develop branch, the tester in that team is responsible for testing that merge request.
\newline


% \noindent From iteration 3 and forward there will be a CFT that is more focused on the testing part, this means that the testers will work more together and are not responsible for specific merge requests. Instead the testers will take on responsibility for a merge request by assigning themselves for testing that merge request in Gitlab.
\noindent From iteration 3 and forward, there will be a CFT that is more focused on the testing part; this means that the testers will work more together and are not responsible for specific merge requests. Instead, the testers will take responsibility for a merge request by assigning themselves to test that merge request in GitLab.
\newline

\noindent \textbf{Test leader}\newline
% \noindent The test leader is responsible for  the  acceptance  testing  and  will coordinate  this  with  the  help  of  all  the  testers.  The  test leader will also be the contact person for non CFT-related inquiries for testing procedures. The test leader will also be responsible for setting up meetings with the customer for performing the acceptance testing.

\noindent The test leader is responsible for the acceptance testing and will coordinate this with the help of all the testers. The test leader will also be the contact person for non-CFT-related inquiries for testing procedures. The test leader will also be responsible for setting up meetings with the customer to perform the acceptance testing.